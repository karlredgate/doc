% vim:expandtab
%
% Copyright (c) 2008 Stratus Technologies, Inc., as an unpublished work.
% This notice does not imply unrestricted or public access to these
% materials which are a trade secret of Stratus Technologies, Inc. or its
% subsidiaries or affiliates (together referred to as "STRATUS"), and
% which may not be reproduced, used, sold or transferred to any third
% party without STRATUS's prior written consent.  All rights reserved.
%
% U.S. Government Rights.  If the Software is to be used by the
% U.S. Government ("Government"), the following provisions apply. The
% Software and associated documentation are commercial computer software
% and commercial computer software documentation, respectively, and the
% Government is acquiring only the license rights granted in the License
% Agreement accompanying this Software (the license rights customarily
% provided to non-Government users), in accordance with Section 12.212
% of the Federal Acquisition Regulations ("FAR") or Section 227.7202-3
% of the Defense Federal Acquisition Regulation Supplement ("DFARS"), as
% applicable. For NASA users, the Government shall obtain the minimum
% rights permitted under Section, in paragraph 1852.227-86 of the NASA
% Supplement to the FAR (or any successor regulations).
%

\ifx\pdfoutput\undefined
\documentclass[twoside]{article}
\else
\documentclass[pdftex,twoside]{article}
\usepackage{thumbpdf}
\pdfcompresslevel=9
\pdfinfo{
   /Author (Karl N. Redgate)
   /Title (Cloud Fast Path from Tervela)
   /Subject (Evaluation)
}
\fi

\usepackage{ifpdf}
% \usepackage{pdftricks}
\usepackage{amssymb}
% \begin{psinputs}
% \usepackage{pstricks}
% \usepackage[all,ps,dvips]{xy}
% \usepackage{pst-node}
% \usepackage{pst-tree}
% \usepackage[dvips]{graphicx}
% \end{psinputs}
\usepackage{moreverb}
\usepackage{wrapfig}
\usepackage{fancyhdr}
\usepackage{makeidx}
% \usepackage{utopia}

\usepackage[pdfcenterwindow=true,pdffitwindow=true,pdfmenubar=true,pdftoolbar=true,bookmarks=true,colorlinks=true]{hyperref}

\newsavebox{\savepar}
\newenvironment{boxit}{\begin{lrbox}{\savepar}\begin{minipage}[b]{6.5in}}
{\end{minipage}\end{lrbox}\fbox{\usebox{\savepar}}}

\newcommand{\vm}{\textsl{virtual machine}\/}
\newcommand{\vms}{\textsl{virtual machines}\/}
\newcommand{\SN}{\textsl{SuperNova}\/}
\newcommand{\harness}{\textsl{test harness}\/}
\newcommand{\WS}{\textsl{WebService}\/}
\newcommand{\spine}{\textsl{spine}\/}
\newcommand{\dendrite}{\textsl{Dendrite}\/}

\newenvironment{note}
{\begin{center}\begin{minipage}{0.8\textwidth}
\vspace{2ex}\hrule\vspace{1ex}
\parskip=\medskipamount
\em}
{\par 
\vspace{1ex}\hrule\vspace{2ex}
\end{minipage}\end{center}}
% End of note environment definition

\pagestyle{fancy}
\fancyfoot[LE,RO]{\thepage}
\fancyfoot[LO,RE]{SuperNova}
\fancyfoot[CO,CE]{\copyright 2008 -- \textsf{Stratus Technologies, Inc.}\\CONFIDENTIAL INFORMATION}

\parskip=\medskipamount

\typeout{Copyright (c) 2008, Stratus Technologies, Inc.}


\title{Cloud Fast Path Evaluation}
\author{Karl N. Redgate}
% \email{Karl.Redgate@gmail.com}

\begin{document}

\maketitle

\begin{abstract}

In order to support the Avance test infrastructure without requiring
IPv4 IP addresses on each node, we need to have an alternative access
method for accessing the devices under test.
Since the existing SuperNova DUTs use IPv6 as a base of their
architecture, we can use IPv6 as the access for the test harness.
\end{abstract}

%Issues:
%\begin{itemize}
%  \setlength{\itemsep}{1pt}
%  \setlength{\parskip}{0pt}
%  \setlength{\parsep}{0pt}
%\item Lab infrastructure
%\item Router
%\item DNS / DHCP / PXE
%\item Test harness
%\item Multiple networks
%\end{itemize}

% ==================================
\section{Overview}

Description of Tervela and their Cloud Fast Path product


\section{Benefits}

suitability for our use

win suggested ``sounds like rsync \+ Molasses \+ jumbo frames''
but then says this might help with first backup etc - jumbo frames will not go over the network

andrew talks about using SCTP for transport on server

http://en.wikipedia.org/wiki/Stream_Control_Transmission_Protocol

User space library
http://www.sctp.de/sctp-download.html

Linux Kernel Stream Control Transmission Protocol Tools
http://lksctp.sourceforge.net/

Better networking with SCTP
http://www.ibm.com/developerworks/library/l-sctp/

http://stackoverflow.com/questions/1171555/why-is-sctp-not-much-used-known

We have been deploying SCTP in several applications now, and encountered
significant problem with SCTP support in various home routers. They simply
don't handle SCTP correctly. I believe this is primarily a performance
issue (the SCTP protocol specification require checksums for the whole
packets to be recalculated and not just for headers).

Like many other promising protocols SCTP is sadly dead in the water
until D-link and Netgear fixes their broken NAT boxes.

http://tools.ietf.org/html/draft-ietf-behave-sctpnat-05



SCTP requires more design within the application to get the best use of
it. There are more options than TCP, the Sockets-like API came later,
and it is young. However I think most people that take the time to
understand it (and who know the shortcomings of TCP) appreciate it --
it is a well designed protocol that builds on our ~30 years of knowledge
of TCP and UDP.

One of the aspects that requires some thought is that of streams. Streams
provide (usually, I think you can turn it off) an order guarantee within
them (much like a TCP connection) but there can be multiple streams per
SCTP connection. If your application's data can be sent over multiple
streams then you avoid head-of-line blocking where the receiver starves
due to one mislaid packet. Effectively different conversations can be
had over the same connection without impacting each other.

Another useful addition is that of multi-homing support -- one connection
can be across multiple interfaces on both ends and it copes with
failures. You can emulate this in TCP, but at the application layer.

Proper link heartbeating, which is the first thing any application using
TCP for non-transient connections implements, is there for free.

My personal summary of SCTP is that it doesn't do anything you couldn't
do another way (in TCP or UDP) with substantial application support. The
thing it provides is the ability to not have to implement that code
(badly) yourself.

FYI, SCTP is mandated as supported for Diameter (cf RADIUS next gen). see
RFC 3588

     Diameter clients MUST support either TCP or SCTP, while agents and
     servers MUST support both.  Future versions of this specification
     MAY mandate that clients support SCTP.




 p1. SCTP mapped directly over IPv4 requires support in NAT gateways,
 which has never been widely deployed anywhere, and without it the typical
 NAT gateway will only permit one private host per public address to be
 using SCTP at a time.

 p2. SCTP mapped over UDP/IPv4 allows more private hosts per public
 address, but UDP mappings in IPv4/NAT gateways are notoriously tricky
 to establish and keep maintained, due to the fact that UDP is a
 connectionless transport without any explicit state for a NAT to track.

 p3. SCTP mapped directly over IPv6 requires\ldots well\ldots
 IPv6. Have you tried to deploy IPv6? If so, have you tried to buy
 an IPv6 firewall? Does it support SCTP? How about a load balancer? A
 SSL accelerator?

 p4. Finally, a lot of the Internet is pretty much constrained to what
 can fit through TCP port 80 and port 443, so SCTP of any flavor tends to
 lose there. Hence, you see efforts like the MPTCP working group in IETF.



 CTP is not very much known and not used/deployed a lot because:

 Widespread: Not widely integrated in TCP/IP stacks (in 2013: still missing natively in latest Mac OSX and Windows)
 Libraries: Few high level bindings in easy to use languages (Disclaimer: i'm maintainer of pysctp, SCTP easy stack support for Python)
 NAT: Doesn't cross NAT very well/at all (less than 1% internet home & enterprise routers do NAT on SCTP).
 Popularity: No general public app use it
 Programming paradigm: it changed a bit: it's still a socket, but you can connect many hosts to many hosts (multihoming), datagram is ordered and reliable, erc\ldots
 Complexity: SCTP stack is complex to implement (due to above)
 Competition: Multipath TCP is coming and should address multihoming needs / capabilities so people refrain from implementing SCTP if possible, waiting for MTCP
 Niche: Needs SCTP fills are very peculiar (ordered reliable datagrams, multistream) and not needed by much applications
 Security: SCTP evades security controls (some firewalls, most IDSes, all DLPs, does not appear on netstat except CentOS/Redhat/Fedora\ldots)
 Audit-ability: Something like 3 companies in the world routinely do audits of SCTP security (Disclaimer: I work in one of them)
 Learning curve: Not much toolchain to play with SCTP (check the excellent withsctp that combines nicely with netcat or use socat )
 Under the hood: Used mostly in telecom and everytime you send SMS, start surfing the net on your mobile or make phone calls, you're often triggering messages that flow over SCTP (SIGTRAN/SS7 with GSM/UMTS, Diameter with LTE/IMS/RCS, S1AP/X2AP with LTE), so you actually use it a lot but you never know about it ;-)



It might not be well known, but it's not unused. Quite recently there was a draft published at the IETF about Using SCTP as a Transport Layer Protocol for HTTP.

http://tools.ietf.org/html/draft-natarajan-http-over-sctp-01



Also STCP??

http://en.wikipedia.org/wiki/Scalable_TCP

http://en.wikipedia.org/wiki/UDP-based_Data_Transfer_Protocol



% ==================================
\begin{thebibliography}{99}

\bibitem{extreme}
   Extreme Networks, Inc.
   \newblock {\em ExtremeXOS Command Reference}.
   \newblock Copyright \copyright~Extreme Networks, Inc., August 2008.\hfil\\
   \newblock {\tt \url{http://www.extremenetworks.com/services/software-userguide.aspx}}

\bibitem{bind9arm}
   \newblock {\em BIND 9 Administrator Reference Manual}.
   \newblock Copyright \copyright~Internet Systems Consortium, Inc., 2008.\hfil\\
   \newblock {\tt \url{https://www.isc.org/files/Bv9.5ARM.pdf}}

\bibitem{rfc2136}
   Paul Vixie
   \newblock {\em Dynamic Updates in the Domain Name System (DNS UPDATE)}.
   \newblock Copyright \copyright~The Internet Society, April 1997.\hfil\\
   \newblock {\tt \url{http://www.ietf.org/rfc/rfc2136.txt}}

\bibitem{uladdr}
   R. Hinden and B. Haberman
   \newblock {\em Unique Local IPv6 Unicast Addresses}.
   \newblock {October 2005}.
   \newblock {\tt \url{http://tools.ietf.org/html/rfc4193}}

\end{thebibliography}

\end{document}

% vim:expandtab
% vim:autoindent
